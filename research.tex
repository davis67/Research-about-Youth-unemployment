

\documentclass[9pt,]{article}
\usepackage{graphicx}%from
\pagenumbering{roman}
\begin{document}
\begin{titlepage}
 \begin{figure}[h]
  \centerline{\small MAKERERE 
  \includegraphics[width=0.1\textwidth]  {muk_log} UNIVERSITY}
\end{figure}
\centerline{COLLEGE OF COMPUTING AND INFORMATION SCIENCES\\}
\paragraph*{•}
\centerline{DEPARTMENT OF COMPUTER SCIENCE\\}
\paragraph*{•}
\centerline{COURSEWORK: RESEARCH METHODOLOGY(BIT 2207)\\}
\paragraph*{•}
\centerline{LECTURER: MR.ERNEST MWEBAZE}
\paragraph*{•}
\centerline{\begin{tabular}{|c|c|c|c|}
\hline
\textbf{No.}& \textbf{Student Name} & \textbf{RegNo} & \textbf{Student Number} \\ \hline
\textit{1}&\textbf{AGABA DAVIS} & \textit{16/U/6236/PS}& \textit{216009915} \\ \hline
\end{tabular}}
\paragraph*{•}
\tableofcontents
\end{titlepage}
\pagenumbering{arabic}
\section{Introduction}
Youth in Uganda are the youngest population in the world, with {77 percent of its population being under 30 years of age.\\In Uganda, the male to female ratio is 100.2 males per 100 females. Life expectancy at birth for males is 42.59 years and 44.49 years for females. Ugandan youth experience different lifestyles depending on if they live in a rural area or urban area. The unemployment rate for young people ages 15–24 is 83 percent.
Analytically, Informal work accounts for the majority of young workers in Uganda.3.2percent of youth work for waged employment, 90.9percent work for informal employment, and 5.8 percent of the Ugandan youth are self-employed.This rate is even higher for those who have formal degrees and live in the urban area.Those without a degree are also not able to obtain jobs because they lack the skills needed for the position or they don’t have the resources such as land or capital.\\Youth unemployment poses a serious political, economic, and social challenge to the country and its leadership. The cycle is making it increasingly difficult for Uganda to break out of poverty. This research is analytical.


\section{State and Structure of Youth Unemployment in Uganda}
For Uganda, in 2012, the Uganda Bureau of Statistics revealed that the share of unemployed youth (national definition, 18-30 years) among the total unemployed persons in the country was 64 percent. Given the rapid growth of the Ugandan population—three-quarters of the population are below the age of 30 years.
According to the International Labor Organization (ILO) definition, Uganda’s measured unemployment  rates are relatively low for the region though they have been increasing from time to time(from 1.9 percent in 2005/06, to 3.6 percent in 2009/10, and recently to 5.1 percent in 2012). At the same time, the characteristics of the unemployed vary widely. Urban youth are more likely to be unemployed (12 percent) than rural youth (3 percent). The above statistics are  used to analyse the problem of youth unemployment using Quantitative research.


\section{Causes of unemployment}
1.	Inadequate investment/supply of jobs
\\2.	insufficient employable skills ie i.e., youth possess skills that are not compatible with available jobs
\\3.	high rates of labor force growth at 4.7% per annum.

\section{Jobs that Ugandan Youth are Engaged in}
Agriculture is the predominant sector of employment in Uganda—providing employment to about 66 percent of the workforce. The services and industrial sectors employ about 28 percent and 7 percent of the labor force, respectively.
Despite the bulk youth employment in agriculture, less than 5 percent of those in agriculture are in wage-paying jobs. The majority are engaged as subsistence family workers with no wages accruing to them. Similarly, informal employment accounts for the highest proportion of employed youths outside agriculture. In 2011, about 95 percent of youth in non-farm enterprises were in informal employment formal jobs are of low-quality, characterized by low and unstable earnings and job insecurity. About 24 percent of employed youth are in wage-paying jobs while the remaining jobs are either in self-employment or household enterprises as contributing workers.
\section{Conclusion}
Interestingly, the report notes that unemployment increases with the level of education attained: Unemployment is lower among persons with no education and primary education, and higher among those with secondary education and above. This is not to negate the importance of education—as it is widely known that education is a significant factor in securing good employment over time—however, the more educated are biased towards wage-paying formal jobs, which are harder to find. Indeed, persons with education above the secondary level are more likely to be in the wage employment(59.1 percent) compared to those with primary education (18 percent), and their earnings tend to be higher.
\section{References}
The Effects of a Very Young Age Structure in Uganda(PDF). Population Action International. Retrieved 5/01/13.
\\ Ugandan Youth Statistics". Retrieved 5/12/13
\\ Ugandan Youth Statistics: Encyclopedia of Urban Ministry UYWI Urban Youth Workers Institute. Urbanministry.org. Retrieved 2013-05-23.\\Byerlee, Derek (Winter 1974). "Rural-Urban Migration in Africa: Theory, Policy and Research Implications". International Migration Review4.\\Bisaso, Ronald (8 July 2010). Organisational responses to public sector reforms in higher education in Uganda: a case study of Makerere University. Journal of Higher Education Policy and Management


\end{document}






